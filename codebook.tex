\documentclass[10pt, a4paper, twocolumn, oneside]{article}

\usepackage[mono]{codebookpkg}

\team{NCTU\_A}{National Chiao Tung University}
\codetheme{default}

\begin{document}

\fontsize{7.7pt}{9.7pt}\selectfont

\tableofcontents

\section{Misc}
    \subsection{Makefile}
        \inputminted{makefile}{content/misc/makefile}
    \subsection{Bump Allocator}
        \cppfile{content/misc/bump-allocator.cpp}

\section{Data Structures}
    \subsection{GNU PBDS}
        \cppfile{content/ds/pbds.cpp}

\section{Math}
    \subsection{Number Theory}
        \subsubsection{Modular}
            \cppfile{content/math/number-theory/modular.cpp}
        \subsubsection{Extended GCD}
            \cppfile{content/math/number-theory/extgcd.cpp}
        \subsubsection{Chinese Remainder}
            \cppfile{content/math/number-theory/crt.cpp}
        \subsubsection{Miller-Rabin}
            \cppfile{content/math/number-theory/miller-rabin.cpp}
        \subsubsection{Tonelli-Shanks}
            \cppfile{content/math/number-theory/tonelli-shanks.cpp}
        \subsubsection{Baby-Step Giant-Step}
            \cppfile{content/math/number-theory/bsgs.cpp}

\section{Numeric}
    \subsection{long long Multiplication}
        \cppfile{content/numeric/ll-mul.cpp}
    \subsection{Barrett Reduction}
        \cppfile{content/numeric/barrett-reduction.cpp}
    \subsection{Fast Fourier Transform}
        \cppfile{content/numeric/fft.cpp}

\section{Graph}
    \subsection{Flow}
        \subsubsection{Dinic}
            \cppfile{content/graph/flow/dinic.cpp}


\end{document}
